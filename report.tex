% Options for packages loaded elsewhere
\PassOptionsToPackage{unicode}{hyperref}
\PassOptionsToPackage{hyphens}{url}
\PassOptionsToPackage{dvipsnames,svgnames,x11names}{xcolor}
%
\documentclass[
  8pt,
  a4paper,
  DIV=11,
  numbers=noendperiod]{scrartcl}

\usepackage{amsmath,amssymb}
\usepackage{iftex}
\ifPDFTeX
  \usepackage[T1]{fontenc}
  \usepackage[utf8]{inputenc}
  \usepackage{textcomp} % provide euro and other symbols
\else % if luatex or xetex
  \usepackage{unicode-math}
  \defaultfontfeatures{Scale=MatchLowercase}
  \defaultfontfeatures[\rmfamily]{Ligatures=TeX,Scale=1}
\fi
\usepackage{lmodern}
\ifPDFTeX\else  
    % xetex/luatex font selection
\fi
% Use upquote if available, for straight quotes in verbatim environments
\IfFileExists{upquote.sty}{\usepackage{upquote}}{}
\IfFileExists{microtype.sty}{% use microtype if available
  \usepackage[]{microtype}
  \UseMicrotypeSet[protrusion]{basicmath} % disable protrusion for tt fonts
}{}
\makeatletter
\@ifundefined{KOMAClassName}{% if non-KOMA class
  \IfFileExists{parskip.sty}{%
    \usepackage{parskip}
  }{% else
    \setlength{\parindent}{0pt}
    \setlength{\parskip}{6pt plus 2pt minus 1pt}}
}{% if KOMA class
  \KOMAoptions{parskip=half}}
\makeatother
\usepackage{xcolor}
\setlength{\emergencystretch}{3em} % prevent overfull lines
\setcounter{secnumdepth}{-\maxdimen} % remove section numbering
% Make \paragraph and \subparagraph free-standing
\makeatletter
\ifx\paragraph\undefined\else
  \let\oldparagraph\paragraph
  \renewcommand{\paragraph}{
    \@ifstar
      \xxxParagraphStar
      \xxxParagraphNoStar
  }
  \newcommand{\xxxParagraphStar}[1]{\oldparagraph*{#1}\mbox{}}
  \newcommand{\xxxParagraphNoStar}[1]{\oldparagraph{#1}\mbox{}}
\fi
\ifx\subparagraph\undefined\else
  \let\oldsubparagraph\subparagraph
  \renewcommand{\subparagraph}{
    \@ifstar
      \xxxSubParagraphStar
      \xxxSubParagraphNoStar
  }
  \newcommand{\xxxSubParagraphStar}[1]{\oldsubparagraph*{#1}\mbox{}}
  \newcommand{\xxxSubParagraphNoStar}[1]{\oldsubparagraph{#1}\mbox{}}
\fi
\makeatother


\providecommand{\tightlist}{%
  \setlength{\itemsep}{0pt}\setlength{\parskip}{0pt}}\usepackage{longtable,booktabs,array}
\usepackage{calc} % for calculating minipage widths
% Correct order of tables after \paragraph or \subparagraph
\usepackage{etoolbox}
\makeatletter
\patchcmd\longtable{\par}{\if@noskipsec\mbox{}\fi\par}{}{}
\makeatother
% Allow footnotes in longtable head/foot
\IfFileExists{footnotehyper.sty}{\usepackage{footnotehyper}}{\usepackage{footnote}}
\makesavenoteenv{longtable}
\usepackage{graphicx}
\makeatletter
\newsavebox\pandoc@box
\newcommand*\pandocbounded[1]{% scales image to fit in text height/width
  \sbox\pandoc@box{#1}%
  \Gscale@div\@tempa{\textheight}{\dimexpr\ht\pandoc@box+\dp\pandoc@box\relax}%
  \Gscale@div\@tempb{\linewidth}{\wd\pandoc@box}%
  \ifdim\@tempb\p@<\@tempa\p@\let\@tempa\@tempb\fi% select the smaller of both
  \ifdim\@tempa\p@<\p@\scalebox{\@tempa}{\usebox\pandoc@box}%
  \else\usebox{\pandoc@box}%
  \fi%
}
% Set default figure placement to htbp
\def\fps@figure{htbp}
\makeatother

\usepackage{etoolbox}
\pretocmd{\tableofcontents}{\clearpage}{}{}
\apptocmd{\tableofcontents}{\clearpage}{}{}
\KOMAoption{captions}{tableheading}
\makeatletter
\@ifpackageloaded{caption}{}{\usepackage{caption}}
\AtBeginDocument{%
\ifdefined\contentsname
  \renewcommand*\contentsname{Table of contents}
\else
  \newcommand\contentsname{Table of contents}
\fi
\ifdefined\listfigurename
  \renewcommand*\listfigurename{List of Figures}
\else
  \newcommand\listfigurename{List of Figures}
\fi
\ifdefined\listtablename
  \renewcommand*\listtablename{List of Tables}
\else
  \newcommand\listtablename{List of Tables}
\fi
\ifdefined\figurename
  \renewcommand*\figurename{Figure}
\else
  \newcommand\figurename{Figure}
\fi
\ifdefined\tablename
  \renewcommand*\tablename{Table}
\else
  \newcommand\tablename{Table}
\fi
}
\@ifpackageloaded{float}{}{\usepackage{float}}
\floatstyle{ruled}
\@ifundefined{c@chapter}{\newfloat{codelisting}{h}{lop}}{\newfloat{codelisting}{h}{lop}[chapter]}
\floatname{codelisting}{Listing}
\newcommand*\listoflistings{\listof{codelisting}{List of Listings}}
\makeatother
\makeatletter
\makeatother
\makeatletter
\@ifpackageloaded{caption}{}{\usepackage{caption}}
\@ifpackageloaded{subcaption}{}{\usepackage{subcaption}}
\makeatother

\usepackage{bookmark}

\IfFileExists{xurl.sty}{\usepackage{xurl}}{} % add URL line breaks if available
\urlstyle{same} % disable monospaced font for URLs
\hypersetup{
  pdftitle={Our Money Report},
  pdfauthor={Talisma Manuel},
  colorlinks=true,
  linkcolor={blue},
  filecolor={Maroon},
  citecolor={Blue},
  urlcolor={Blue},
  pdfcreator={LaTeX via pandoc}}


\title{Our Money Report}
\author{Talisma Manuel}
\date{report of 18 July, 2025}

\begin{document}
\maketitle

\renewcommand*\contentsname{Table of contents}
{
\hypersetup{linkcolor=}
\setcounter{tocdepth}{3}
\tableofcontents
}

\subsection{Gestao dos Gastos}\label{gestao-dos-gastos}

\begin{figure}

\begin{minipage}{0.50\linewidth}

\subsubsection{Onde estou gastando mais
dinheiro?}\label{onde-estou-gastando-mais-dinheiro}

\pandocbounded{\includegraphics[keepaspectratio]{report_files/figure-pdf/cell-3-output-1.pdf}}

\end{minipage}%
%
\begin{minipage}{0.50\linewidth}

\subsubsection{Quanto gastei no total nos ultimos
meses?}\label{quanto-gastei-no-total-nos-ultimos-meses}

Neste mês: \textbf{14,514.68}

Nos últimos 6 meses \textbf{104,414.62}

\subsubsection{Quais foram minhas maiores despesas esse
mes?}\label{quais-foram-minhas-maiores-despesas-esse-mes}

Categorias com gasto acima de 1000:

\begin{longtable}[]{@{}ll@{}}
\caption{}\label{T_520e8}\tabularnewline
\toprule\noalign{}
Categoria & Total \\
\midrule\noalign{}
\endfirsthead
\toprule\noalign{}
Categoria & Total \\
\midrule\noalign{}
\endhead
\bottomrule\noalign{}
\endlastfoot
alimentaçao & 2590.48 \\
aluguel & 3100.00 \\
angola & 2916.00 \\
kixikila & 2000.00 \\
saude & 1296.50 \\
\end{longtable}

O maior gasto este mês foi em: \textbf{aluguel} com \textbf{3,100.00}

\end{minipage}%

\end{figure}%

\begin{figure}

\begin{minipage}{0.50\linewidth}

\subsubsection{Existem despesas recorrentes nos trimeste / simestre
?}\label{existem-despesas-recorrentes-nos-trimeste-simestre}

ultimo trimeste\end{minipage}%
%
\begin{minipage}{0.50\linewidth}

\end{minipage}%
\newline
\begin{minipage}{0.50\linewidth}

\begin{longtable}[]{@{}lll@{}}
\caption{}\label{T_ae836}\tabularnewline
\toprule\noalign{}
category & Total & Percentual (\%) \\
\midrule\noalign{}
\endfirsthead
\toprule\noalign{}
category & Total & Percentual (\%) \\
\midrule\noalign{}
\endhead
\bottomrule\noalign{}
\endlastfoot
alimentaçao & 9583.58 & 17.62 \\
aluguel & 9300.00 & 17.10 \\
angola & 4016.00 & 7.38 \\
transporte & 3325.00 & 6.11 \\
kixikila & 3100.00 & 5.70 \\
bebida & 2274.80 & 4.18 \\
documentos & 2258.50 & 4.15 \\
educaçao & 2249.00 & 4.13 \\
convivio & 2160.86 & 3.97 \\
emprestimo & 2050.00 & 3.77 \\
casa & 2038.80 & 3.75 \\
saude & 1846.50 & 3.39 \\
gastos extra & 1326.28 & 2.44 \\
restaurante & 1231.00 & 2.26 \\
geovana & 1092.00 & 2.01 \\
roupas & 1033.00 & 1.90 \\
roupa & 950.00 & 1.75 \\
internet & 747.00 & 1.37 \\
aparelhos & 596.00 & 1.10 \\
presente & 518.00 & 0.95 \\
estetica & 510.00 & 0.94 \\
divida & 411.00 & 0.76 \\
babysitting & 400.00 & 0.74 \\
agua e luz & 345.90 & 0.64 \\
gastos extras & 217.00 & 0.40 \\
mala & 180.00 & 0.33 \\
lazer & 176.00 & 0.32 \\
saldo & 129.00 & 0.24 \\
gas & 100.00 & 0.18 \\
presentes & 99.00 & 0.18 \\
calçados & 75.00 & 0.14 \\
ajuda & 20.00 & 0.04 \\
prensentes & 20.00 & 0.04 \\
higiene pessoal & 13.00 & 0.02 \\
ayla & 5.00 & 0.01 \\
\end{longtable}

\end{minipage}%
%
\begin{minipage}{0.50\linewidth}
ultimo semestre

\begin{longtable}[]{@{}lll@{}}
\caption{}\label{T_cab28}\tabularnewline
\toprule\noalign{}
category & Total & Percentual (\%) \\
\midrule\noalign{}
\endfirsthead
\toprule\noalign{}
category & Total & Percentual (\%) \\
\midrule\noalign{}
\endhead
\bottomrule\noalign{}
\endlastfoot
aluguel & 18600.00 & 17.73 \\
alimentaçao & 17318.01 & 16.50 \\
angola & 6866.00 & 6.54 \\
transporte & 6534.00 & 6.23 \\
educaçao & 6277.00 & 5.98 \\
aparelhos & 5645.00 & 5.38 \\
roupas & 4802.85 & 4.58 \\
emprestimo & 4420.00 & 4.21 \\
saude & 3786.20 & 3.61 \\
geovana & 3697.00 & 3.52 \\
documentos & 3458.50 & 3.30 \\
kixikila & 3100.00 & 2.95 \\
casa & 3010.80 & 2.87 \\
gastos extra & 2767.41 & 2.64 \\
bebida & 2665.10 & 2.54 \\
convivio & 2160.86 & 2.06 \\
internet & 1495.00 & 1.42 \\
restaurante & 1231.00 & 1.17 \\
babysitting & 1200.00 & 1.14 \\
roupa & 950.00 & 0.91 \\
agua e luz & 836.89 & 0.80 \\
presentes & 721.00 & 0.69 \\
estetica & 680.00 & 0.65 \\
presente & 518.00 & 0.49 \\
divida & 517.00 & 0.49 \\
saldo & 262.00 & 0.25 \\
gastos extras & 217.00 & 0.21 \\
gas & 200.00 & 0.19 \\
mala & 180.00 & 0.17 \\
yachane & 180.00 & 0.17 \\
lazer & 176.00 & 0.17 \\
talisma & 139.00 & 0.13 \\
netflix & 130.00 & 0.12 \\
calçados & 75.00 & 0.07 \\
higiene pessoal & 43.00 & 0.04 \\
ajuda & 25.00 & 0.02 \\
ayla & 25.00 & 0.02 \\
prensentes & 20.00 & 0.02 \\
\end{longtable}

\end{minipage}%
\newline
\begin{minipage}{0.50\linewidth}

\end{minipage}%

\end{figure}%

\subsubsection{Meu padrao de gasto mudou comprado ao meses anteriores
?}\label{meu-padrao-de-gasto-mudou-comprado-ao-meses-anteriores}

\pandocbounded{\includegraphics[keepaspectratio]{report_files/figure-pdf/cell-9-output-1.pdf}}

\subsubsection{Como meus gastos em evoluíram ao longo do
tempo?}\label{como-meus-gastos-em-evoluuxedram-ao-longo-do-tempo}

\pandocbounded{\includegraphics[keepaspectratio]{report_files/figure-pdf/cell-10-output-1.pdf}}

\pandocbounded{\includegraphics[keepaspectratio]{report_files/figure-pdf/cell-10-output-2.pdf}}

\pandocbounded{\includegraphics[keepaspectratio]{report_files/figure-pdf/cell-10-output-3.pdf}}

\pandocbounded{\includegraphics[keepaspectratio]{report_files/figure-pdf/cell-10-output-4.pdf}}

\pandocbounded{\includegraphics[keepaspectratio]{report_files/figure-pdf/cell-10-output-5.pdf}}

\pandocbounded{\includegraphics[keepaspectratio]{report_files/figure-pdf/cell-10-output-6.pdf}}

\pandocbounded{\includegraphics[keepaspectratio]{report_files/figure-pdf/cell-10-output-7.pdf}}

\pandocbounded{\includegraphics[keepaspectratio]{report_files/figure-pdf/cell-10-output-8.pdf}}

\subsection{Receitas}\label{receitas}

\subsubsection{Quais são minhas principais fontes de
receita?}\label{quais-suxe3o-minhas-principais-fontes-de-receita}

\begin{longtable}[]{@{}llll@{}}
\caption{}\label{T_c6047}\tabularnewline
\toprule\noalign{}
Categoria & Total & Porcentagem (\%) & Total \\
\midrule\noalign{}
\endfirsthead
\toprule\noalign{}
Categoria & Total & Porcentagem (\%) & Total \\
\midrule\noalign{}
\endhead
\bottomrule\noalign{}
\endlastfoot
salario & 271996.00 & 89.20 & 271996.00 \\
seguro & 11985.00 & 3.90 & 11985.00 \\
emprestimo & 10741.00 & 3.50 & 10741.00 \\
kixikila & 5200.00 & 1.70 & 5200.00 \\
bolsa amci & 2500.00 & 0.80 & 2500.00 \\
investimento & 1200.00 & 0.40 & 1200.00 \\
gastos extra & 900.00 & 0.30 & 900.00 \\
extra & 240.00 & 0.10 & 240.00 \\
divida & 100.00 & 0.00 & 100.00 \\
geovana & 50.00 & 0.00 & 50.00 \\
\end{longtable}

\subsubsection{Minha receita total aumentou ou diminuiu nos últimos
meses?}\label{minha-receita-total-aumentou-ou-diminuiu-nos-uxfaltimos-meses}

\pandocbounded{\includegraphics[keepaspectratio]{report_files/figure-pdf/cell-12-output-1.pdf}}

\subsection{Orçamento e Metas}\label{oruxe7amento-e-metas}

\begin{figure}

\begin{minipage}{0.50\linewidth}

\subsubsection{Em quais categorias de orçamento estou estourando o
limite com mais
frequência?}\label{em-quais-categorias-de-oruxe7amento-estou-estourando-o-limite-com-mais-frequuxeancia}

\begin{longtable}[]{@{}ll@{}}
\caption{}\label{T_9319c}\tabularnewline
\toprule\noalign{}
Categoria & Percentual de Estouro (\%) \\
\midrule\noalign{}
\endfirsthead
\toprule\noalign{}
Categoria & Percentual de Estouro (\%) \\
\midrule\noalign{}
\endhead
\bottomrule\noalign{}
\endlastfoot
alimentaçao & 27.78 \\
roupas & 80.00 \\
bebida & 73.33 \\
agua e luz & 100.00 \\
dividas & 0.00 \\
saude & 57.14 \\
\end{longtable}

\end{minipage}%
%
\begin{minipage}{0.50\linewidth}

\subsubsection{Quanto ainda tenho disponível para gastar em cada
categoria neste
mês?}\label{quanto-ainda-tenho-disponuxedvel-para-gastar-em-cada-categoria-neste-muxeas}

\begin{longtable}[]{@{}ll@{}}
\caption{}\label{T_bf9e5}\tabularnewline
\toprule\noalign{}
categoria & restou \\
\midrule\noalign{}
\endfirsthead
\toprule\noalign{}
categoria & restou \\
\midrule\noalign{}
\endhead
\bottomrule\noalign{}
\endlastfoot
saude & -796.50 \\
bebida & -49.85 \\
dividas & 0.00 \\
agua e luz & 100.00 \\
roupas & 200.00 \\
alimentaçao & 409.52 \\
\end{longtable}

\end{minipage}%

\end{figure}%

\subsubsection{Qual o meu progresso em relação à minha meta de
economia?}\label{qual-o-meu-progresso-em-relauxe7uxe3o-uxe0-minha-meta-de-economia}

\begin{longtable}[]{@{}llllllllllll@{}}
\caption{}\label{T_65ab0}\tabularnewline
\toprule\noalign{}
jan & fev & mar & abr & mai & jun & jul & ago & set & out & nov & dez \\
\midrule\noalign{}
\endfirsthead
\toprule\noalign{}
jan & fev & mar & abr & mai & jun & jul & ago & set & out & nov & dez \\
\midrule\noalign{}
\endhead
\bottomrule\noalign{}
\endlastfoot
- & 5016.30 & -2754.86 & 4814.48 & 258.03 & -16631.50 & 18576.00 &
-9593.78 & 4133.22 & 5376.90 & 2051.68 & -11465.19 \\
7066.12 & 5212.37 & 14390.28 & -7048.82 & -1102.17 & 629.75 & -8314.68 &
- & - & - & - & - \\
\end{longtable}

\subsubsection{Quanto estou conseguindo economizar em média por
mês?}\label{quanto-estou-conseguindo-economizar-em-muxe9dia-por-muxeas}

Média de economia nos últimos 3 meses: \textbf{-2929.03}

\subsubsection{Estou conseguindo seguir meu orçamento nesse
mes?}\label{estou-conseguindo-seguir-meu-oruxe7amento-nesse-mes}

\pandocbounded{\includegraphics[keepaspectratio]{report_files/figure-pdf/cell-18-output-1.pdf}}




\end{document}
